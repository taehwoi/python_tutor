\documentclass{beamer}
\usepackage{inconsolata}
\usepackage{graphicx}
\usepackage{caption}
\usepackage{color}
\usepackage{listings}
\usepackage{kotex}
\usepackage{subfig}
\usepackage{cooltooltips}
%\usepackage{epsfig, graphics}
\usepackage{hyperref}
\usepackage{perpage}
\usepackage[normalem]{ulem}
\setbeamertemplate{navigation symbols}{}%remove navigation symbols
\usepackage{listings}
\usepackage{color}
\usepackage{framed}

\definecolor{background}{RGB}{39, 40, 34}
\definecolor{string}{RGB}{230, 219, 116}
\definecolor{comment}{RGB}{117, 113, 94}
\definecolor{normal}{RGB}{248, 248, 242}
\definecolor{identifier}{RGB}{166, 226, 46}



\lstset{
  language=C,               			% choose the language of the code
  alsolanguage=Python,            			% choose the language of the code
  alsolanguage=Java,            			% choose the language of the code
  numbers=none,                   		% where to put the line-numbers
  stepnumber=1,                   		% the step between two line-numbers.        
  numbersep=5pt,                  		% how far the line-numbers are from the code
  extendedchars=true,
  numberstyle=\scriptsize\color{black}\ttfamily,
  backgroundcolor=\color{background},  		% choose the background color. You must add \usepackage{color}
  showspaces=false,               		% show spaces adding particular underscores
  showstringspaces=false,         		% underline spaces within strings
  showtabs=false,                 		% show tabs within strings adding particular underscores
  frame=single,
  framerule=0pt,
  tabsize=4,                      		% sets default tabsize to 2 spaces
  captionpos=n,                   		% sets the caption-position to bottom
  breaklines=true,                		% sets automatic line breaking
  breakatwhitespace=true,         		% sets if automatic breaks should only happen at whitespace
  title=\lstname,                 		% show the filename of files included with \lstinputlisting;
  basicstyle=\color{normal}\scriptsize\ttfamily,					% sets font style for the code
  keywordstyle=\color{magenta}\scriptsize\ttfamily,	% sets color for keywords
  stringstyle=\color{string}\scriptsize\ttfamily,		% sets color for strings
  commentstyle=\color{comment}\scriptsize\ttfamily,	% sets color for comments
  emph={True, False, format_string, eff_ana_bf, permute, eff_ana_btr},
  emphstyle=\color{identifier}\scriptsize\ttfamily,
  morekeywords={with, as}
}

\lstset{literate=%
   *{0}{{{\color{cyan}0}}}1
    {1}{{{\color{cyan}1}}}1
    {2}{{{\color{cyan}2}}}1
    {3}{{{\color{cyan}3}}}1
    {4}{{{\color{cyan}4}}}1
    {5}{{{\color{cyan}5}}}1
    {6}{{{\color{cyan}6}}}1
    {7}{{{\color{cyan}7}}}1
    {8}{{{\color{cyan}8}}}1
    {9}{{{\color{cyan}9}}}1
}



\newenvironment{enum}{
\begin{enumerate}
  \setlength{\itemsep}{1pt}
  \setlength{\parskip}{0pt}
  \setlength{\parsep}{0pt}
}{\end{enumerate}}


\hypersetup{
  colorlinks=true,
  urlcolor=pink,
}

\MakePerPage{footnote}

\title{Python 101}
\subtitle{Lec -1\\ 되돌아보기}
\author{엄태휘}

\begin{document}
\frame{\titlepage}

\begin{frame}
\frametitle{되돌아보기}
\textit{"한 학기를 되돌아보는 제일 좋은 방법은 기말고사입니다."}\\
-K모 교수-
\end{frame}

\begin{frame}
\frametitle{문제 0/10}
제 이름은?
\end{frame}

\begin{frame}[fragile]
\frametitle{문제 1/10}
다음 코드를 시행한 결과는?
\begin{lstlisting}
a = "deadbeef"
print(len((set(a))))
\end{lstlisting}
\end{frame}

\begin{frame}[fragile]
\frametitle{문제 2/10}
Python의 dict의 원소에서 대해서 일반적으로 $O(1)$의 접근을 할 수 있는 이유는?
\begin{enumerate}
  \item hash
  \item back propagation
  \item hershey
  \item gradient descent
\end{enumerate}
\end{frame}

\begin{frame}[fragile]
\frametitle{문제 3/10}
중간계산결과를 저장하여 중복되는 계산을 줄이는 기법의 이름은?
\begin{enumerate}
  \item loop unrolling
  \item garbage collection
  \item memoization
  \item sensation
\end{enumerate}
\end{frame}

\begin{frame}[fragile]
\frametitle{문제 4/10}
다음 코드의 시간복잡도 $O(?)$를 말하시오.
\begin{lstlisting}
def f(arr):
    if len(arr) <= 1:
        return arr
    pivot = arr[len(arr) // 2]
    lesser_arr, equal_arr, greater_arr = [], [], []
    for num in arr:
        if num < pivot:
            lesser_arr.append(num)
        elif num > pivot:
            greater_arr.append(num)
        else:
            equal_arr.append(num)
    return f(lesser_arr) + equal_arr + f(greater_arr)
\end{lstlisting}
\end{frame}

\begin{frame}
농담이고\\
(궁금하시다면\ldots평균: $O(NlogN)$, 최악:$O(N^2)$)
\end{frame}

\begin{frame}[fragile]{스타트업 이야기를 좀 해보아요}
\begin{lstlisting}
print('START-UP'.araboza())
\end{lstlisting}
\end{frame}

\begin{frame}[fragile]{스타트업 이야기를 좀 해보아요}
Q: 좋은 스타트업은 도대체 뭔가요?
\end{frame}

\begin{frame}[fragile]{물어보았습니다}
\begin{figure}[H]
  \centering
  \subfloat{
    \includegraphics[width=0.45\linewidth]{k1.png}}
  \subfloat{
    \includegraphics[width=0.45\linewidth]{k2.png}}
\end{figure}
\end{frame}

\begin{frame}[fragile]{물어보았습니다}
\begin{figure}[H]
  \includegraphics[width=0.6\linewidth]{kakao2.png}
\end{figure}
\end{frame}

\begin{frame}[fragile]{물어보았습니다}
\begin{figure}[H]
  \centering
  \subfloat{
    \includegraphics[width=0.45\linewidth]{k3.png}}
  \subfloat{
    \includegraphics[width=0.45\linewidth]{k4.png}}
\end{figure}
\end{frame}

\begin{frame}[fragile]{물어보았습니다}
\begin{figure}[H]
  \centering
  \subfloat{
    \includegraphics[width=0.45\linewidth]{kakao3_5.png}}
\end{figure}
\end{frame}

\begin{frame}[fragile]{물어보았습니다}
\begin{figure}[H]
  \centering
  \subfloat{
    \includegraphics[width=0.45\linewidth]{kk5.png}}
  \subfloat{
    \includegraphics[width=0.45\linewidth]{kk6.png}}
\end{figure}
\end{frame}

\begin{frame}[fragile]{물어보았습니다}
\begin{figure}[H]
  \centering
    \includegraphics[width=0.65\linewidth]{kakao5.png}
\end{figure}
\end{frame}

\begin{frame}[fragile]{물어보았습니다}
\begin{figure}[H]
  \centering
    \includegraphics[width=0.65\linewidth]{kakao7.png}
\end{figure}
\end{frame}

\begin{frame}[fragile]{물어보았습니다}
\begin{figure}[H]
  \centering
    \includegraphics[width=0.45\linewidth]{./insta.jpg}
\end{figure}
\end{frame}

\begin{frame}[fragile]{물어보았습니다}
\begin{figure}[H]
  \centering
  \subfloat{
    \includegraphics[width=0.45\linewidth]{kakao9.png}}
  \subfloat{
    \includegraphics[width=0.45\linewidth]{kakao8.png}}
\end{figure}
\end{frame}

\begin{frame}[fragile]{스타트업 이야기를 좀 해보아요}
Q: 스타트업에 일하고 싶은 이유는 어떻게 되시나요?
\end{frame}

\begin{frame}[fragile]{다시 파이선으로}
프로그래밍 연습하기 좋은 곳:\\
https://www.acmicpc.net/\\
프로젝트 기반 학습:\\
https://github.com/tuvtran/project-based-learning\#python
\end{frame}

\begin{frame}[fragile]
  \centering 감사합니다\\

\end{frame}



\end{document}
