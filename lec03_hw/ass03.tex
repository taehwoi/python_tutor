\documentclass{article}
\usepackage{bm}
\usepackage{amsmath}
\usepackage{graphicx}
\usepackage{mdwlist}
\usepackage[colorlinks=true]{hyperref}
\usepackage{geometry}
\geometry{margin=1in}
\geometry{headheight=2in}
\geometry{top=1.0in}
\usepackage{palatino}
\usepackage{ulem}
\usepackage[parfill]{parskip}
%\renewcommand{\rmdefault}{palatino}
\usepackage{fancyhdr}
\usepackage{listings}
\usepackage{color}
\usepackage{framed}

\definecolor{background}{RGB}{39, 40, 34}
\definecolor{string}{RGB}{230, 219, 116}
\definecolor{comment}{RGB}{117, 113, 94}
\definecolor{normal}{RGB}{248, 248, 242}
\definecolor{identifier}{RGB}{166, 226, 46}



\lstset{
  language=C,               			% choose the language of the code
  alsolanguage=Python,            			% choose the language of the code
  alsolanguage=Java,            			% choose the language of the code
  numbers=none,                   		% where to put the line-numbers
  stepnumber=1,                   		% the step between two line-numbers.        
  numbersep=5pt,                  		% how far the line-numbers are from the code
  extendedchars=true,
  numberstyle=\scriptsize\color{black}\ttfamily,
  backgroundcolor=\color{background},  		% choose the background color. You must add \usepackage{color}
  showspaces=false,               		% show spaces adding particular underscores
  showstringspaces=false,         		% underline spaces within strings
  showtabs=false,                 		% show tabs within strings adding particular underscores
  frame=single,
  framerule=0pt,
  tabsize=4,                      		% sets default tabsize to 2 spaces
  captionpos=n,                   		% sets the caption-position to bottom
  breaklines=true,                		% sets automatic line breaking
  breakatwhitespace=true,         		% sets if automatic breaks should only happen at whitespace
  title=\lstname,                 		% show the filename of files included with \lstinputlisting;
  basicstyle=\color{normal}\scriptsize\ttfamily,					% sets font style for the code
  keywordstyle=\color{magenta}\scriptsize\ttfamily,	% sets color for keywords
  stringstyle=\color{string}\scriptsize\ttfamily,		% sets color for strings
  commentstyle=\color{comment}\scriptsize\ttfamily,	% sets color for comments
  emph={True, False, format_string, eff_ana_bf, permute, eff_ana_btr},
  emphstyle=\color{identifier}\scriptsize\ttfamily,
  morekeywords={with, as}
}

\lstset{literate=%
   *{0}{{{\color{cyan}0}}}1
    {1}{{{\color{cyan}1}}}1
    {2}{{{\color{cyan}2}}}1
    {3}{{{\color{cyan}3}}}1
    {4}{{{\color{cyan}4}}}1
    {5}{{{\color{cyan}5}}}1
    {6}{{{\color{cyan}6}}}1
    {7}{{{\color{cyan}7}}}1
    {8}{{{\color{cyan}8}}}1
    {9}{{{\color{cyan}9}}}1
}


%\pagestyle{fancy}

\newcommand{\infint}{\int_{-\infty}^{\infty}}
\newcommand{\Ab}{\bm{A}}
\rhead{}
\lhead{}
\chead{%
  {\vbox{%
      \vspace{1mm}
      \large
      Python practice 3\hfill \\
    }
  }
}

\usepackage{paralist}

\usepackage{todonotes}
\setlength{\marginparwidth}{2.15cm}

\usepackage{tikz}
\usetikzlibrary{positioning,shapes,backgrounds}

\newenvironment{enum}{
\begin{enumerate}
  \setlength{\itemsep}{1pt}
  \setlength{\parskip}{0pt}
  \setlength{\parsep}{0pt}
}{\end{enumerate}}

\begin{document}
\thispagestyle{fancy}
\setcounter{section}{-1}

%% Q0
\section{Recursion \& Fuction Practice: Sum}
Write a recursive function $my\_sum(lst)$ that sums up a list.
Check that it works just like python's $sum(lst)$.\\

\textit{hint 1}: Assume $my\_sum(shorter\_list)$ somehow gives you the correct value
magically. Then what?

\textit{hint 2}: $assert(boolean)$ would raise
AssertionError if the condition is False, and nothing happens if the condition
is True.

\begin{lstlisting}
def my_sum(lst):
    # NOTE: sum([]) == 0
    if len(lst) == 0:
      return 0
    else:
      # FILL ME IN

l = # some random list
assert(my_sum(l) == sum(l))
\end{lstlisting}

%% Q1
\section{Recursion \& Fuction Practice: Max}
Write a recursive function $my\_max(lst)$ that returns maximum value of a
list.
Check that it works just like python's $max(lst)$.
\begin{lstlisting}
def my_max(lst):
    # NOTE: max([]) is not defined.
    if len(lst) == 1:
      return lst[0]
    else:
      # FILL ME IN

l = # some random list
assert(my_max(l) == max(l))
\end{lstlisting}

%% Q2
\section{Recursion \& Fuction Practice: Binary to Decimal}
Write a recursive function $bin\_to\_dec(binary)$ that returns value of a
binary number(represented as a string) evaluted to a decimal number.
Check that it works just like python's $int(binary,\ 2).$(surprised?)

\textit{hint 1}: $int(binary,\ 2)$, $int(ternary,\ 3)$ $int(hexadecimal,\
16)$\ldots\ you name it. The default is $int(n,\ 10)$.
\begin{lstlisting}
def bin_to_dec(binary):
    if binary == '0':
      return 0
    elif binary == '1':
      return 1
    else:
      # FILL ME IN

binary_string = input()
print(bin_to_dec(binary_string) == int(binary_string, 2))
\end{lstlisting}

%% Q3
\section{Recursion \& Fuction Practice: Reverse}
Write a recursive $reverse(lst)$ that returns the reversed list.
\begin{lstlisting}
def reverse(lst):
    #FILL_ME_IN

l = # random list
assert(reverse(l) == l[::-1])
\end{lstlisting}

%% Q4
\section{Recursion \& Fuction Practice: Is Palindrome?}
Write a recursive $is\_palindrome(s)$ that returns True if a string is
palindrome, and False otherwise.

\textit{hint}: The following recursive definition of a palindrome would help.
\begin{alignat*}{5}
  &\text{P}&\quad\rightarrow&&\text{"" (empty)}\\
  &\quad&\mid\ &&\quad\text{c + P + c}\quad&&
\end{alignat*}
\begin{lstlisting}
def is_palindrome(s):
    #FILL_ME_IN

s = input()
print(is_palindrome(s))
\end{lstlisting}

\section{Partitioning}
  Calculate the number of cases of partitioning number N with 1,2,3.\\
  \begin{align*}
  4 &= 1+1+1+1\\
    &= 1+1+2\\
    &= 1+2+1\\
    &= 1+3\\
    &= 2+1+1\\
    &= 2+2\\
    &= 3+1
  \end{align*}
\begin{lstlisting}
4
7
\end{lstlisting}
\textit{Optional}: Can it calculate numbers like 100? Try memoization.

\section{Binary Search}
Remember the Up\&Down games during the parties?
A participant would guess a number in $range(1, 51)$, and the other player
would tell you if the answer is lower, or higher than the guess.
The optimal strategy would be to call the median (25$\rightarrow$12 if down, 37
if up), that halves
the search space everytime.

Binary Search is an algorithm that does exactly this. Write a recursive
$bin\_search(lst, n)$, finds n from an already sorted list.
It would return True if n is in our list, and False
otherwise.

\textit{hint}: If the median of the list is the value we are looking for, return True.
Else, we will have to look at the right half, or the left half.

\begin{lstlisting}
def bin_search(lst, n):
    # FILL_ME_IN


l = sorted(#some random list)
print(bin_search(l, n))
\end{lstlisting}

\section{The idea behind dynamic programming}
Given a list of integers, finding its partial sum is
easy: $sum(lst[start:end])$. However, if we were to calculate the partial sum
100,000,000 times, it would approximately take $(end-start) * 100,000,000$
times to calculate it.

Write a function $partial\_sum(start, end)$ that calculates
$sum(lst[start:end])$ fast, by storing some data in advance.

\textit{hint}: $sum(lst[start:end])$ can be expressed as the difference of
certain two values.


%% Q3
\section{Prefix Calculator}
We usually write mathematical expressions this way:
$3+4*2$. This is called an infix notation, and although easy to understand, it has
some drawbacks.
\begin{enumerate}
  \item We have to know the precedence of operators to correctly compute the
    expression. Without knowing that '*' comes before '+', $3+4*2$ can be
    miscalculated as $14$.
  \item We have to use parentheses to express the intended order of operations.
    $(3+4)*2$ if we want the result to be 14.
\end{enumerate}
Therefore, there are other notations like prefix notations(operators precede
their operands), or postfix notations(operators follow their operands).
For example, the expression above would be $(+\ 3\ (*\ 4\
2))$ in prefix notations.

To evaluate this expression, we start from the left, and when we see an
operator('+'), we expect two sub-expressions. We see $3$ and (* 4 2).
3 is 3. For (* 4 2), we repeat the step above, with an operator('*') and
two sub-expressions $4$ and $2$. Since there are no more sub-expressions to be
evaluated, we calculate $(*\ 4\ 2) = 8$, and $(+\ 3\ 8) = 11$.

TLDR; we are going to implement a mathematical expression calculator.
For your convenience, I have already implemented the $infix\_to\_prefix(exp)$
(There is no need to understand the code).
It's your job to fill in $calculate(exp)$ that evaluates the prefix notation,
\textit{recursively}.


\textit{INPUT}: A single line of a correct mathematical expression, with or
without parentheses.

\textit{OUTPUT}: The result of calculating the mathematical expression.

\textit{CONDITION}: For simplicity, the numbers used in the expression are
single digit, ranging from 0 to 9. The operators used are $+(addition)$,
$-(substraction)$, $*(multiplication)$, and $/(division)$.

\textit{EXAMPLE}:
\begin{lstlisting}
(3+2)*2
10

3+2*2
7
\end{lstlisting}

\textit{hint}: Before implementing $calculate(exp)$, examine the prefix notation
of expressions. We can see that the prefix expression is formed as following(we
call this structure the tree structure):
\begin{alignat*}{3}
  exp &\rightarrow \ \  &&(operand,\ exp,\ exp)\ \  &&\text{: tuple}\\
      &\ \ \ | &&\quad\quad\quad\quad n &&\text{: integer}
\end{alignat*}
When the type of the exp is integer, (in Python:
$isinstance(prefix\_exp,\ int)$), there
is nothing to do; just return $n$.

When it's a tuple, we have to do more. Again, assuming that
$calculate(smaller\_exp)$ will magically calculate the results would help.\\

\section{Challenge: DP and optimization}
DP and optimization go along well, because an optimized solution of smaller
problem often lead to an optimized solution of the whole.

Consider the following situation; We want to travel to Europe(\textit{or any
country you like}) $N$ days from now, and we are going to work part-time for
the tickets. It would be great if we can work every single day, but we have to
take a rest - we can't work 3 consecutive days.

Given a list of $N$ integers, each representing the money we can earn,
calculate the maximum amount of money we can earn.

\begin{lstlisting}
10 20 30 40 50 60 
160
\end{lstlisting}

\textit{hint}: We can work 20,30,50,60 to earn 160. This is larger than any
other combination($e.g.$ 10,20,40,50). One approach would be a recursive one.
Let $solution(d)\text{ = maximum money earned while working on $\text{d}^{th}$ day}$.
Then we can calculate $solution(d)$ by comparing two diffrent cases.
One would be adding $\text{d}^{th}$ and $\text{d-1}^{th}$ payroll with the maximum money we earned 3 days ago,
the other would be adding $\text{d}^{th}$ payroll with the maximum money we earned 2
days ago.


%% Q4
\section{Challenge: Decompressing DNAs}
DISCLAIMER: I am not a biology major so if there is anything wrong here, please
forgive/tell me.

DNAs are composed of $a,g,c,t$(am I right?). They are very long(am I right?).
Therefore, there is ongoing research on compressing DNA sequences for data
efficieny. One method of compressing DNAs utilizes the fact that common
patterns occur in a DNA strain (am I right?).

We can compress a DNA sequence by writing out the repeated pattern
in the following manner: $K(S)$. K is a $single$ digit integer indicating the number of
repetition, and S is the repeated sequence, which can be compressed as well.

For example, $A3(CGT)3(A)2(3(C)GT)$ could be decompressed as\\
$A-CGT-CGT-CGT-AAA-CCCGT-CCCGT$ (- added for readability).

Write a code that prints the DNA sequence, given the compressed version.\\
\textit{hint}: You might want to locate the index of '(' by s.find(), and find the
matching ')' by counting the number of '('s and ')'s inside. Using recursion
would help.

\begin{lstlisting}
A3(CGT)3(A)2(3(C)GT)
ACGTCGTCGTAAACCCGTCCCGT

3(3(A)G)
AAAGAAAGAAAG
\end{lstlisting}



\end{document}
