\documentclass{article}
\usepackage{bm}
\usepackage{amsmath}
\usepackage{graphicx}
\usepackage{mdwlist}
\usepackage{kotex}
\usepackage[colorlinks=true]{hyperref}
\usepackage{geometry}
\geometry{margin=1in}
\geometry{headheight=2in}
\geometry{top=1.0in}
\usepackage{palatino}
\usepackage{ulem}
\usepackage[parfill]{parskip}
%\renewcommand{\rmdefault}{palatino}
\usepackage{fancyhdr}
\usepackage{listings}
\usepackage{color}
\usepackage{framed}

\definecolor{background}{RGB}{39, 40, 34}
\definecolor{string}{RGB}{230, 219, 116}
\definecolor{comment}{RGB}{117, 113, 94}
\definecolor{normal}{RGB}{248, 248, 242}
\definecolor{identifier}{RGB}{166, 226, 46}



\lstset{
  language=C,               			% choose the language of the code
  alsolanguage=Python,            			% choose the language of the code
  alsolanguage=Java,            			% choose the language of the code
  numbers=none,                   		% where to put the line-numbers
  stepnumber=1,                   		% the step between two line-numbers.        
  numbersep=5pt,                  		% how far the line-numbers are from the code
  extendedchars=true,
  numberstyle=\scriptsize\color{black}\ttfamily,
  backgroundcolor=\color{background},  		% choose the background color. You must add \usepackage{color}
  showspaces=false,               		% show spaces adding particular underscores
  showstringspaces=false,         		% underline spaces within strings
  showtabs=false,                 		% show tabs within strings adding particular underscores
  frame=single,
  framerule=0pt,
  tabsize=4,                      		% sets default tabsize to 2 spaces
  captionpos=n,                   		% sets the caption-position to bottom
  breaklines=true,                		% sets automatic line breaking
  breakatwhitespace=true,         		% sets if automatic breaks should only happen at whitespace
  title=\lstname,                 		% show the filename of files included with \lstinputlisting;
  basicstyle=\color{normal}\scriptsize\ttfamily,					% sets font style for the code
  keywordstyle=\color{magenta}\scriptsize\ttfamily,	% sets color for keywords
  stringstyle=\color{string}\scriptsize\ttfamily,		% sets color for strings
  commentstyle=\color{comment}\scriptsize\ttfamily,	% sets color for comments
  emph={True, False, format_string, eff_ana_bf, permute, eff_ana_btr},
  emphstyle=\color{identifier}\scriptsize\ttfamily,
  morekeywords={with, as}
}

\lstset{literate=%
   *{0}{{{\color{cyan}0}}}1
    {1}{{{\color{cyan}1}}}1
    {2}{{{\color{cyan}2}}}1
    {3}{{{\color{cyan}3}}}1
    {4}{{{\color{cyan}4}}}1
    {5}{{{\color{cyan}5}}}1
    {6}{{{\color{cyan}6}}}1
    {7}{{{\color{cyan}7}}}1
    {8}{{{\color{cyan}8}}}1
    {9}{{{\color{cyan}9}}}1
}


%\pagestyle{fancy}

\newcommand{\infint}{\int_{-\infty}^{\infty}}
\newcommand{\Ab}{\bm{A}}
\rhead{}
\lhead{}
\chead{%
  {\vbox{%
      \vspace{1mm}
      \large
      Python practice 2.5\hfill \\
    }
  }
}

\usepackage{paralist}

\usepackage{todonotes}
\setlength{\marginparwidth}{2.15cm}

\usepackage{tikz}
\usetikzlibrary{positioning,shapes,backgrounds}

\newenvironment{enum}{
\begin{enumerate}
  \setlength{\itemsep}{1pt}
  \setlength{\parskip}{0pt}
  \setlength{\parsep}{0pt}
}{\end{enumerate}}

\begin{document}
\thispagestyle{fancy}
\setcounter{section}{-1}

%% Q1
\section{Look and say sequence}
  Print the $N^{th}$ look and say sequence, a.k.a ant sequence.\\
  1, 11, 12, 1121, 122111....
  The sequence goes like this:\\
  1$\rightarrow$11(1이 
  1개)$\rightarrow$12(1이 2개)$\rightarrow$1121(1이 1개, 2가 1개)$\rightarrow$122111(1이 2개, 2가 1개, 1이 1개)\\
\textit{EXAMPLE}:
\begin{lstlisting}
3
12
\end{lstlisting}
%\noindent\rule{\textwidth}{0.9pt}

%% Q2
\section{Prime?}
Given input $N$($2\le N \le 10,000,000,000$), Print 'True' if $N$ is a prime number. Print 'False' otherwise. 
\textit{EXAMPLE}:
\begin{lstlisting}
1217
True
\end{lstlisting}

%% Q3
\section{Wrd Shrtnr}
Given input, remove vowels(\textit{a,e,i,o,u}).\\
\textit{EXAMPLE}:
\begin{lstlisting}
chicken
chckn
\end{lstlisting}

%% Q4
\section{Digit Counter}
Given N, print the occurrence of each digit without using the Counter
class.(The answer is similar to how Counter works)\\
\textit{EXAMPLE}:
\begin{lstlisting}
12353
0:0
1:1
2:1
3:2
4:0
5:1
6:0
7:0
8:0
9:0
\end{lstlisting}

\section{Where are you?}
We can easily get the maximum value of a list by $max(lst)$.\\
But finding its index is not as easy. Write a program that prints the index of
the maximum value. (Note that the index starts from 0).
\begin{lstlisting}
10 3 5 100 9
3
\end{lstlisting}

\section{Genetic Algorithm}
Genetic Algorithm is an optimization technique that mimics the process of
evolution and natural selection. This time, we are going to find the largest
number in between ($0_{(2)}, 1111111111_{(2)}$) using GA. The answer
is obviously $1111111111_{(2)}$, but lets see if natural selection can figure that out.
Please fill in the $TODO$s. Reading this link would help:
\href{https://namu.wiki/w/유전알고리즘}{유전알고리즘}.\\
Download "genetic\_skelton.py" to start.

\end{document}
