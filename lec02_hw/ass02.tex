\documentclass{article}
\usepackage{bm}
\usepackage{amsmath}
\usepackage{graphicx}
\usepackage{mdwlist}
\usepackage{kotex}
\usepackage[colorlinks=true]{hyperref}
\usepackage{geometry}
\geometry{margin=1in}
\geometry{headheight=2in}
\geometry{top=1.0in}
\usepackage{palatino}
\usepackage{ulem}
\usepackage[parfill]{parskip}
%\renewcommand{\rmdefault}{palatino}
\usepackage{fancyhdr}
\usepackage{listings}
\usepackage{color}
\usepackage{framed}

\definecolor{background}{RGB}{39, 40, 34}
\definecolor{string}{RGB}{230, 219, 116}
\definecolor{comment}{RGB}{117, 113, 94}
\definecolor{normal}{RGB}{248, 248, 242}
\definecolor{identifier}{RGB}{166, 226, 46}



\lstset{
  language=C,               			% choose the language of the code
  alsolanguage=Python,            			% choose the language of the code
  alsolanguage=Java,            			% choose the language of the code
  numbers=none,                   		% where to put the line-numbers
  stepnumber=1,                   		% the step between two line-numbers.        
  numbersep=5pt,                  		% how far the line-numbers are from the code
  extendedchars=true,
  numberstyle=\scriptsize\color{black}\ttfamily,
  backgroundcolor=\color{background},  		% choose the background color. You must add \usepackage{color}
  showspaces=false,               		% show spaces adding particular underscores
  showstringspaces=false,         		% underline spaces within strings
  showtabs=false,                 		% show tabs within strings adding particular underscores
  frame=single,
  framerule=0pt,
  tabsize=4,                      		% sets default tabsize to 2 spaces
  captionpos=n,                   		% sets the caption-position to bottom
  breaklines=true,                		% sets automatic line breaking
  breakatwhitespace=true,         		% sets if automatic breaks should only happen at whitespace
  title=\lstname,                 		% show the filename of files included with \lstinputlisting;
  basicstyle=\color{normal}\scriptsize\ttfamily,					% sets font style for the code
  keywordstyle=\color{magenta}\scriptsize\ttfamily,	% sets color for keywords
  stringstyle=\color{string}\scriptsize\ttfamily,		% sets color for strings
  commentstyle=\color{comment}\scriptsize\ttfamily,	% sets color for comments
  emph={True, False, format_string, eff_ana_bf, permute, eff_ana_btr},
  emphstyle=\color{identifier}\scriptsize\ttfamily,
  morekeywords={with, as}
}

\lstset{literate=%
   *{0}{{{\color{cyan}0}}}1
    {1}{{{\color{cyan}1}}}1
    {2}{{{\color{cyan}2}}}1
    {3}{{{\color{cyan}3}}}1
    {4}{{{\color{cyan}4}}}1
    {5}{{{\color{cyan}5}}}1
    {6}{{{\color{cyan}6}}}1
    {7}{{{\color{cyan}7}}}1
    {8}{{{\color{cyan}8}}}1
    {9}{{{\color{cyan}9}}}1
}


%\pagestyle{fancy}

\newcommand{\infint}{\int_{-\infty}^{\infty}}
\newcommand{\Ab}{\bm{A}}
\rhead{}
\lhead{}
\chead{%
  {\vbox{%
      \vspace{1mm}
      \large
      Python practice 2\hfill \\
    }
  }
}

\usepackage{paralist}

\usepackage{todonotes}
\setlength{\marginparwidth}{2.15cm}

\usepackage{tikz}
\usetikzlibrary{positioning,shapes,backgrounds}

\newenvironment{enum}{
\begin{enumerate}
  \setlength{\itemsep}{1pt}
  \setlength{\parskip}{0pt}
  \setlength{\parsep}{0pt}
}{\end{enumerate}}

\begin{document}
\thispagestyle{fancy}
\setcounter{section}{-1}

%% Q1
\section{Star Starry Night~★}
%\noindent\rule{\textwidth}{0.9pt}
\textit{EXAMPLE}:
\begin{lstlisting}
3
*
**
***
\end{lstlisting}
%\noindent\rule{\textwidth}{0.9pt}

%% Q2
\section{Starry Starry Night~★}
\textit{EXAMPLE}:
\begin{lstlisting}
3
*****
 ***
  *
\end{lstlisting}

\section{99 dan}
Print out the following:
\begin{lstlisting}
1 * 1 = 1
1 * 2 = 2
   ...
9 * 9 = 81
\end{lstlisting}
We \textit{can} construct the string by concatanation like this:
\begin{lstlisting}
str(8) + ' * ' + str(9) + ' = ' + str(72)
\end{lstlisting}
but there is a better way.
\href{https://www.google.com/search?q=python+string+format}{Google It!}

%% Q3
\section{Lotto}
Create and print a list of length \textit{6}, where each element is a random
integer in range $[1, 45]$.

\begin{lstlisting}
lst = # Do Something
print(*lst)
\end{lstlisting}

To learn how to generate random numbers
\href{https://www.google.com/search?q=python+random+number}{Google It!}
When we use \textit{list comprehension}, we can do this with a single line of code.


\section{Buying Lotto}
Now, get input from user(us) $until$ the user has correctly guessed the 7 numbers
created above, to become a millionaire.\\

\section{2nd}
Create a list of 100 random numbers.\\
$OUTPUT$: the 2nd largest number from the list.\\
The following solution is pretty obvious, but try solving it $without$ sorting the list.
\begin{lstlisting}
print(sorted(lst[-2]))
\end{lstlisting}

\section{Acronymizer}
Given a series of words, print its acronym. An acronym should be all capital
letters, with $'.'$ in between the characters.
\textit{EXAMPLE}:
\begin{lstlisting}
come late and start sleeping
C.L.A.S.S.
\end{lstlisting}

\section{esrever}
In Python, we can create a reversed list(or any iterable), by
$iterable[::-1]$. However, this can be memory inefficient as it creates a new
iterable. Write a code that reverses a list without creating a new list.\\
\textit{hint}: The following code swaps the value of two variables.
\begin{lstlisting}
x = 0
y = 1
x, y = y, x
print(x, y)
\end{lstlisting}

\section{Fibonacci}
The Fibonacci numbers are defined as following:
$$F_0 = 0, F_1 = 1$$
$$F_n = F_{n-1}+F_{n-2}, n > 1$$
Input: N, Output: $N^th$ Fibonacci.\\
Check that this program also prints $F_0$ and $F_1$ well, for input 0 and 1.\\
\end{document}
