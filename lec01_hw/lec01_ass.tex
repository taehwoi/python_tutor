\documentclass{article}
\usepackage{bm}
\usepackage{amsmath}
\usepackage{graphicx}
\usepackage{mdwlist}
\usepackage[colorlinks=true]{hyperref}
\usepackage{geometry}
\geometry{margin=1in}
\geometry{headheight=2in}
\geometry{top=1.0in}
\usepackage{palatino}
%\usepackage{ulem}
\usepackage[parfill]{parskip}
%\renewcommand{\rmdefault}{palatino}
\usepackage{fancyhdr}
\usepackage{listings}
\usepackage{color}
\usepackage{framed}

\definecolor{background}{RGB}{39, 40, 34}
\definecolor{string}{RGB}{230, 219, 116}
\definecolor{comment}{RGB}{117, 113, 94}
\definecolor{normal}{RGB}{248, 248, 242}
\definecolor{identifier}{RGB}{166, 226, 46}



\lstset{
  language=C,               			% choose the language of the code
  alsolanguage=Python,            			% choose the language of the code
  alsolanguage=Java,            			% choose the language of the code
  numbers=none,                   		% where to put the line-numbers
  stepnumber=1,                   		% the step between two line-numbers.        
  numbersep=5pt,                  		% how far the line-numbers are from the code
  extendedchars=true,
  numberstyle=\scriptsize\color{black}\ttfamily,
  backgroundcolor=\color{background},  		% choose the background color. You must add \usepackage{color}
  showspaces=false,               		% show spaces adding particular underscores
  showstringspaces=false,         		% underline spaces within strings
  showtabs=false,                 		% show tabs within strings adding particular underscores
  frame=single,
  framerule=0pt,
  tabsize=4,                      		% sets default tabsize to 2 spaces
  captionpos=n,                   		% sets the caption-position to bottom
  breaklines=true,                		% sets automatic line breaking
  breakatwhitespace=true,         		% sets if automatic breaks should only happen at whitespace
  title=\lstname,                 		% show the filename of files included with \lstinputlisting;
  basicstyle=\color{normal}\scriptsize\ttfamily,					% sets font style for the code
  keywordstyle=\color{magenta}\scriptsize\ttfamily,	% sets color for keywords
  stringstyle=\color{string}\scriptsize\ttfamily,		% sets color for strings
  commentstyle=\color{comment}\scriptsize\ttfamily,	% sets color for comments
  emph={True, False, format_string, eff_ana_bf, permute, eff_ana_btr},
  emphstyle=\color{identifier}\scriptsize\ttfamily,
  morekeywords={with, as}
}

\lstset{literate=%
   *{0}{{{\color{cyan}0}}}1
    {1}{{{\color{cyan}1}}}1
    {2}{{{\color{cyan}2}}}1
    {3}{{{\color{cyan}3}}}1
    {4}{{{\color{cyan}4}}}1
    {5}{{{\color{cyan}5}}}1
    {6}{{{\color{cyan}6}}}1
    {7}{{{\color{cyan}7}}}1
    {8}{{{\color{cyan}8}}}1
    {9}{{{\color{cyan}9}}}1
}


%\pagestyle{fancy}

\newcommand{\infint}{\int_{-\infty}^{\infty}}
\newcommand{\Ab}{\bm{A}}
\rhead{}
\lhead{}
\chead{%
  {\vbox{%
      \vspace{1mm}
      \large
      Python practice 1\hfill \\
    }
  }
}

\usepackage{paralist}

\usepackage{todonotes}
\setlength{\marginparwidth}{2.15cm}

\usepackage{tikz}
\usetikzlibrary{positioning,shapes,backgrounds}

\newenvironment{enum}{
\begin{enumerate}
  \setlength{\itemsep}{1pt}
  \setlength{\parskip}{0pt}
  \setlength{\parsep}{0pt}
}{\end{enumerate}}

\begin{document}
\thispagestyle{fancy}
\setcounter{section}{-1}

%% Q1
\section{Dumb Multiplication}
When I was a kid, I used to multiply numbers in the following manner:
$27 \times 33 = 621$.\\
This calculation is done by concatanating the product of
each digits.($2\times3=6$ \& $7\times3=21$).\\
Given a 2-digit integer $N$ and $K$, print the result of dumb multiplying $N$ and $K$.\\
\noindent\rule{\textwidth}{0.9pt}
\textit{CONDITION}: $ 10 \le N, K \le 99 $\\
\textit{EXAMPLE}:\\
\begin{lstlisting}
27 33

621
\end{lstlisting}

\section{Sigma Calculator}
Given $N$ and $K$, calculate the following:
$$\sum_{i=N}^{K} i$$\\
Please refrain from doing something like:
\begin{lstlisting}
print(K*(K-1)/2 - (N-1)*(N-2)/2) 
\end{lstlisting}
although this is a faster and better solution.\\
\noindent\rule{\textwidth}{0.9pt}
\textit{CONDITION}: $N \le K$\\
\textit{EXAMPLE}:\\
\begin{lstlisting}
2 10

54
\end{lstlisting}

\section{Numbers Numbers Numbers}
Given 3 integers, print their median and mean. When printing the mean, print up
to the $2^{nd}$ decimal digit. We can do it via format printing.
\href{https://www.google.com/search?q=python+format+float+length}{Here's how}

\textit{EXAMPLE}:\\
\begin{lstlisting}
3 3 3

3 3.00
\end{lstlisting}

\begin{lstlisting}
1 2 17

2 6.66
\end{lstlisting}

\section{369}
For positive interger $n$, if $n$ is a multiple of $3$ ($e.g.$ $3,6,9...$) or
contains $3$ ($e.g.$ 13, 23...), print \textit{CLAP}.\\
Else, print \textit{SAFE}.\\
\noindent\rule{\textwidth}{0.9pt}
\textit{hint}:
\begin{lstlisting}
# % is the modulus operator
print(17 % 2) # == 1
print(20 % 2) # == 0
\end{lstlisting}
\textit{EXAMPLE}:\\
\begin{lstlisting}
31
CLAP
\end{lstlisting}
\begin{lstlisting}
5
SAFE
\end{lstlisting}

\section{Anagrams}
Given 2 strings, print \textit{True} if they are anagrams of each other, and
print \textit{False} otherwise.\\
Note: We should ignore whitespaces, and capitalization does not matter.\\
\noindent\rule{\textwidth}{0.9pt}
\textit{hint}:
\begin{lstlisting}
s = "abc def"
# creates a new string where whitespace(' ') is replaced by an empty string('')
# strings are immutable, so .replace() creates a new string
a = s.replace(' ', '')
print(a)
\end{lstlisting}
\textit{EXAMPLE}:\\
\begin{lstlisting}
Dormitory
Dirty Room

True
\end{lstlisting}
\begin{lstlisting}
Eleven plus two
Twelve plus one

True
\end{lstlisting}
\begin{lstlisting}
Soju
Gobchang

False
\end{lstlisting}

\pagebreak
\section{YootNori}
YootNori's result is determined by the number of \textit{Bae}s(represented as
$0$) and \textit{Deung}s(represented as $1$) of four yoots:\\
$Do$ (1 $Bae$, 3 $Deung$s), $Gae$ (2 $Bae$s, 2 $Deung$s), $Gul$ (3 $Bae$s, 1
$Deung$), $Yoot$ (4 $Bae$s), $Moe$(4 $Deungs$).\\
From the given combination of $Bae$s and $Deung$s, determine the result.\\
\noindent\rule{\textwidth}{0.9pt}
\textit{EXAMPLE}:\\
\begin{lstlisting}
0
0
0
0

Yoot
\end{lstlisting}
\begin{lstlisting}
0
1
1
0

Gae
\end{lstlisting}
\end{document}
