\documentclass{beamer}
\usepackage{inconsolata}
\usepackage{color}
\usepackage{listings}
\setbeamertemplate{navigation symbols}{}%remove navigation symbols
\usepackage{listings}
\usepackage{color}
\usepackage{framed}

\definecolor{background}{RGB}{39, 40, 34}
\definecolor{string}{RGB}{230, 219, 116}
\definecolor{comment}{RGB}{117, 113, 94}
\definecolor{normal}{RGB}{248, 248, 242}
\definecolor{identifier}{RGB}{166, 226, 46}



\lstset{
  language=C,               			% choose the language of the code
  alsolanguage=Python,            			% choose the language of the code
  alsolanguage=Java,            			% choose the language of the code
  numbers=none,                   		% where to put the line-numbers
  stepnumber=1,                   		% the step between two line-numbers.        
  numbersep=5pt,                  		% how far the line-numbers are from the code
  extendedchars=true,
  numberstyle=\scriptsize\color{black}\ttfamily,
  backgroundcolor=\color{background},  		% choose the background color. You must add \usepackage{color}
  showspaces=false,               		% show spaces adding particular underscores
  showstringspaces=false,         		% underline spaces within strings
  showtabs=false,                 		% show tabs within strings adding particular underscores
  frame=single,
  framerule=0pt,
  tabsize=4,                      		% sets default tabsize to 2 spaces
  captionpos=n,                   		% sets the caption-position to bottom
  breaklines=true,                		% sets automatic line breaking
  breakatwhitespace=true,         		% sets if automatic breaks should only happen at whitespace
  title=\lstname,                 		% show the filename of files included with \lstinputlisting;
  basicstyle=\color{normal}\scriptsize\ttfamily,					% sets font style for the code
  keywordstyle=\color{magenta}\scriptsize\ttfamily,	% sets color for keywords
  stringstyle=\color{string}\scriptsize\ttfamily,		% sets color for strings
  commentstyle=\color{comment}\scriptsize\ttfamily,	% sets color for comments
  emph={True, False, format_string, eff_ana_bf, permute, eff_ana_btr},
  emphstyle=\color{identifier}\scriptsize\ttfamily,
  morekeywords={with, as}
}

\lstset{literate=%
   *{0}{{{\color{cyan}0}}}1
    {1}{{{\color{cyan}1}}}1
    {2}{{{\color{cyan}2}}}1
    {3}{{{\color{cyan}3}}}1
    {4}{{{\color{cyan}4}}}1
    {5}{{{\color{cyan}5}}}1
    {6}{{{\color{cyan}6}}}1
    {7}{{{\color{cyan}7}}}1
    {8}{{{\color{cyan}8}}}1
    {9}{{{\color{cyan}9}}}1
}



\title{Python 101}
\subtitle{Lec01 \\ Python and Primitive Data Types}
\author{thoum}

\begin{document}
\frame{\titlepage}

\begin{frame}
\frametitle{WHY Python?}
\begin{itemize}
    \item
    Hello, World! in C
    \begin{lstinputlisting}
      {./hello.c}
    \end{lstinputlisting}
    \item
    Hello, World! in JAVA
    \begin{lstinputlisting}
      {./hello.java}
    \end{lstinputlisting}
\end{itemize}
\end{frame}

\begin{frame}
\frametitle{WHY Python?}
\begin{itemize}
    \item
      Hello, World! in
      Brainfuck\footnote{\small{https://namu.wiki/w/BrainFuck}}
    \begin{lstinputlisting}
      {./hello.bf}
    \end{lstinputlisting}
\end{itemize}
\end{frame}

\begin{frame}
\frametitle{WHY Python?}
  \textcolor{black}{\LARGE{However in Python..}}
\end{frame}
\begin{frame}
    \begin{lstinputlisting}
      {./hello.py}
    \end{lstinputlisting}
\end{frame}

\begin{frame}
\frametitle{WHY Python?}
  {\Large{Python lets you\ldots}}
  \begin{itemize}
    \item Care less about the system details.
    \item Think at the high level.
  \end{itemize}
\end{frame}

\begin{frame}
\frametitle{WHY Python?}
  \begin{center}
  {\Large{Lets Begin!}}
  \end{center}

\end{frame}

\begin{frame}
  \frametitle{The Built-ins}
    \begin{lstinputlisting}
      {./primitive_types.py}
    \end{lstinputlisting}
\end{frame}

\begin{frame}
  \frametitle{Numerals}
    \begin{lstinputlisting}[firstline=1, lastline=6]
      {./primitive_types.py}
    \end{lstinputlisting}
\end{frame}

\begin{frame}
  \frametitle{Numerals}
  \textcolor{black}{Yes, the numbers we know.}
  \begin{itemize}
    \item Python has \alert{UNLIMITED} precision for integers.

      (C's intmax is usually 2147483647)
    \item There are errors present in floats.
  \end{itemize}

\end{frame}

\begin{frame}
  \textcolor{black}{\large{Possible Operations with Integers}}
    \begin{lstinputlisting}
      {./int_ops.py}
    \end{lstinputlisting}
\end{frame}

\begin{frame}
  \frametitle{Strings}
    \begin{lstinputlisting}[firstline=8, lastline=12]
      {./primitive_types.py}
    \end{lstinputlisting}

\end{frame}

\begin{frame}
  \frametitle{Strings}
  \begin{itemize}
    \item Strings are enclosed in matching (') or (")s.(No difference)
    \item Note that a single character is also a string.
  \end{itemize}
\end{frame}

\begin{frame}
  \frametitle{Strings}
  \textcolor{black}{\large{Possible Operations with Strings}}

  There are many things that can be done with a string(or an
  \textit{ITERABLE}), but we deal with them later.
  \begin{lstinputlisting}
    {./string_ops.py}
  \end{lstinputlisting}
\end{frame}

\begin{frame}
  \frametitle{Booleans}
  True and False.
  \begin{lstinputlisting}[firstline=14, lastline=15]
    {./primitive_types.py}
  \end{lstinputlisting}
\end{frame}

\begin{frame}
  \frametitle{Strings}
  \textcolor{black}{\large{Possible Operations with Booleans}}
  \begin{lstinputlisting}
    {./bool_ops.py}
  \end{lstinputlisting}
\end{frame}

\begin{frame}[plain, c]
  \begin{center}
    Questions?
  \end{center}
\end{frame}

\begin{frame}{Iterables}
  We can combine the data types in lists, tuples\ldots
\end{frame}


\end{document}
